\documentclass{article}

\begin{document}
\title{User Manual for Accurity}
\date{Jan 30th, 2017}
\author{Yao Su, Zhihui Luo, Yu S. Huang}
\maketitle


\section{Introduction}
Accurity is a software that infers tumor purity and ploidy from a pair of tumor-normal genome sequencing data.


Z. Luo*, X. Fan*, Y.Su, YS. Huang (2018). Accurity: Accurate tumor purity and ploidy inference from tumor-normal WGS data by jointly modelling somatic copy number alterations and heterozygous germline single-nucleotide-variants. Bioinformatics.

Website: http://www.yfish.org/software

\section{License}
\subsection{FOR-PROFIT PURPOSES}
If you plan to use Accurity in any for-profit application, you are required to purchase a license. To do so, please contact yuhuang@simm.ac.cn to discuss your application.

\subsection{ACADEMIC NON-COMMERCIAL RESEARCH PURPOSES}
Under this license (included in the software release), you can use the program for free as long as you are using Accurity strictly for non-profit purposes.

\section{Install}
\subsection{Prerequisites}

R, Python2, SAMtools. Python package \textbf{pyflow}, https://github.com/Illumina/pyflow, and R packages \textbf{ggplot2}, \textbf{grid}, \textbf{scales}, \textbf{glmnet}, \textbf{MASS}, \textbf{RColorBrewer} and \textbf{colorspace}.

Running Accurity requires a project-specific \textbf{configure} file, details below. Change the path of R and samtools in file \textbf{configure} according to your OS environment.

\subsection{Unpack Accurity.tar.gz}

Uncompress the tarball by
\ \\
\indent \textbf{tar -xvzf Accurity.tar.gz} \\
\label{sec_aux}

Accurity is a package that contains a few binary executables, and R/Python scripts. All binary executables were compiled for a Linux platform (ubuntu 14 and 16 tested). It also contains a sample \textbf{configure} file. Denote the software full path as \textit{Accurity\_path} in file \textbf{configure}(described below).

\subsubsection{A folder of reference data files}
From the website above, there is another download, containing a directory \textit{refData/} of size 13G. subfolder \textit{refData/hs37d5/} contains GC-counting binary files about the reference genome, used for the GC-correction step. Subfolder \textit{refData/} contains common (allele frequency $>$5\%) SNPs from the 1000 Genome project. All these reference data is based on reference genome version hs37d5. We included the reference genome (the original hs37d5 and hs37d5 with only chromosome 1 to 22) in the tarball. Users can contact us if you need GC-counting binary files of a different reference genome.

Folder \textit{refData/} can be be put anywhere you like. Denote the full path as \textit{path\_to\_reference\_data}.

\section{Project organization and file \textbf{configure}}

\subsection{Project organization}

For an example, you have a project directory \textbf{project\_A} with a pair of matched tumor and normal samples. \\

\indent \textbf{project\_A/} \\
\indent \indent \textbf{sample\_1\_cancer.bam} \\
\indent \indent \textbf{sample\_1\_cancer.bam.bai} \\
\indent \indent \textbf{sample\_1\_normal.bam} \\
\indent \indent \textbf{sample\_1\_normal.bam.bai} \\


\textbf{*.bam.bai} files (bam index) are not required. Accurity will use samtools to generate them if not present.


\subsection{Setup the configure file}

Copy the sample \textbf{configure} file from the Accurity package into your project folder and modify it accordingly. An example looks like this:

\begin{verbatim}
project_name    CRC
reference_human_genome  hs37d5
readlength_(bp) 101
window_size_(bp)        500
path_to_reference_data  /home/user/Accurity/refData
path_to_reference_genome_data   /y/hlab/AccurityTestData/simulation/hs37d5_namechr.fa
full_path_to_samtools   /y/program/bin/samtools
full_path_to_freebayes  /y/program/bin/freebayes
Accurity_path   /y/program/cancer_purity/Accurity
\end{verbatim}


Meaning of the fields in the \textbf{configure} file:

\begin{description}
  \item[reference\_genome\_name] the version of the reference genome to which
  the bam files are aligned to. It is assumed that all samples under the same directory are aligned to the same reference genome.
  \item[read\_length] the length in base pair of the read
  \item[window\_size] the window size in base pair for segmentation. The segmentation program (BIC-seq) first calculates the number of reads for each window and then perform segmentation over the genome. A small window size often leads to a large number of small segments. The recommended window size is 500bp.
  \item[reference\_index\_folder\_path] the directory where the reference genome index data (pre-generated by accurity) is stored. See section \ref{sec_aux}.
  \item[reference\_genome\_fasta\_path] path to the reference genome fasta file
  \item[samtools\_path] path to the samtools program.
  \item[freebayes\_path] path to the freebayes program.
  \item[accurity\_path] path to the Accurity software. See section \ref{sec_code}.
\end{description}

\subsection{Run Accurity}
Accurity is composed of several C++ binaries. To make it easy to run, we have written a Python wrapper, \textbf{main.py}, that wraps all binary executables in a pipeline.


Get help:

\textbf{./main.py -h} \\

Run Accurity from scratch, given two bam files and an output folder \textbf{sample\_1\_infer}:

\textbf{./main.py -c path\_to\_configure\_file -t sample\_1\_cancer.bam -n sample\_1\_normal.bam -o sample\_1\_infer }\\

Resume Accurity from a specific step:

\textbf{./main.py -c path\_to\_configure\_file -t sample\_1\_cancer.bam -n sample\_1\_normal.bam -o sample\_1\_infer \textit{ $-$s S}}

Step \textit{ S} has five choices, 0 to 4, with 0 equivalent to running from the very beginning. The series of steps are: 0 Generate directory and bam index files; 1 Get read locations and heterozygous germline SNVs; 2 Correction for GC bias; 3. Segment the genome; 4 Infer purity and ploidy. Given a specific \textit{ S},  all steps larger than or equal to \textit{ S} will be carried out.\\

Change the default lambda value for BIC-seq:

\textbf{./main.py -c path\_to\_configure\_file -t sample\_1\_cancer.bam -n sample\_1\_normal.bam -o sample\_1\_infer  \textit{ $-$l lambda}}

\textit{ lambda} is an integer parameter for BIC-seq, controlling the sensitivity of segmentation. Default is 4.\\

Override previous results:

\textbf{./main.py -c path\_to\_configure\_file -t sample\_1\_cancer.bam -n sample\_1\_normal.bam -o sample\_1\_infer \textit{ $--$clean 1}}

\subsection{Accurity output}
After running Accurity, the directory structure will look like:\\

\indent \textbf{project\_A/} \\
\indent \indent \textit{sample\_1\_infer/} \\
\indent \indent \textit{sample\_1\_cancer/} \\
\indent \indent \textbf{sample\_1\_cancer.bam} \\
\indent \indent \textbf{sample\_1\_cancer.bam.bai} \\
\indent \indent \textit{sample\_1\_normal/} \\
\indent \indent \textbf{sample\_1\_normal.bam} \\
\indent \indent \textbf{sample\_1\_normal.bam.bai} \\

The italic folders are output folders created for each bam file. All major results are stored in the output directory \textit{sample\_1\_infer}. File \textit{sample\_1\_infer/infer.out.tsv} contains the output. An example looks like: \\

\begin{verbatim}
purity  ploidy  depth
0.10256 2.5     4.9707
logL    period  best_no_of_copy_nos_bf_1st_peak first_peak_int
2.3657e+07      50      2       975
\end{verbatim}


There are two cases where purity and ploidy can not be inferred. 1) The cancer genome contains too few somatic copy number alterations; 2) The noise level is too high, or the noise level is moderate but the sample purity is very low ($<$0.05).


\end{document}
